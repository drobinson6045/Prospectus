%% LyX 2.2.1 created this file.  For more info, see http://www.lyx.org/.
%% Do not edit unless you really know what you are doing.
\documentclass[12pt,english]{article}
\usepackage[T1]{fontenc}
\usepackage[latin9]{inputenc}
\usepackage{geometry}
\geometry{verbose,tmargin=2cm,bmargin=2cm,lmargin=2cm,rmargin=2cm}
\usepackage{fancyhdr}
\pagestyle{fancy}
\PassOptionsToPackage{normalem}{ulem}
\usepackage{ulem}
\usepackage{babel}
\begin{document}

\section{Introduction}

alsdkjf

\section{Interface Tracking Model}

Wildfires are multiscale systems with complex dynamics and feedback
mechanisms, making it difficult to capture all of their characteristics.
Physics based models, such as HIGRAD/FIRETEC {[}4{]} (H/F), model
the fluid dynamics, the fire-atmosphere interaction, and the chemical
processes (pyrolysis) of fires. These models can simulate a wide range
of conditions but have large computational requirements, making them
unviable as an operational model. This project aims to cast the modeling
of fire as an interface tracking problem where a fireline will be
represented as either a single closed curve (entire interior burning)
,$\Gamma_{0}$, or as an outer fireline with one or multiple inner
firelines (regions of exhausted or unburnt fuel), $\Gamma_{i}$ for
$i\geq1$. The firelines, $\Gamma_{i},$ represent the isotherm encapsulating
the heavy burning regime. This is a non-local and semi-empirical model
where normal velocities, $\vec{v_{n}},$ are derived from a combination
of heat flux from combustion, convective sinks, and fire-atmosphere
interactions. 

\subsection{Heat Flux}

The heat flux is calculated as an integral over the area of the enclosed
burning region. 

\subsection{Fire-Atmosphere Interactions}

The fire-atmosphere interactions will be determined by coupling the
Weather Research and Forecasting (WRF) model {[}5{]} with the changes
in air pressure resulting from heat flux. All quantities calculated
along the interface are done in Fourier space, resulting in minimal
computational cost and high-order accuracy. Parameters for the model
will be determined through Bayesian Inference with the Delayed Rejection
Adaptive Metropolis (DRAM) algorithm {[}1{]} using data collected
from experiments carried out at Tall Timbers Research Station (TT)
and model runs from H/F. Other factors that need further consideration
are merging of fronts and fuel bed properties. \textbf{\small{}The
goal is to create an operational wildland fire model that captures
fireline behaviors that have been observed in other models and experiments
at a much smaller computational cost. }{\small \par}

\textbf{\uline{Intellectual Merit}}\textbf{:} A numerical issue
with modeling a fireline as a moving interface is the distortion of
the mesh as the interface moves. This leads to a varying spatial resolution
and can cause tangling and instabilities. By introducing a tangential
velocity component {[}2{]} it is guaranteed that points remain equispaced
in arclength without affecting the shape of the interface (Figure
\ref{lTheta}). With this formulation, a well-behaved mesh is attained
and a large number of stable timesteps can be taken. This framework
makes this model a novel approach to modeling wildfires.

\section{Model Fitting}

\section{SARndbox}

\section{Conclusion}
\end{document}
